\section{Preliminary \label{sec-def}}

A heterogeneous information network is a special kind of information network, which either contains multiple types of objects or multiple types of links.
\begin{myDef}
\textbf{Heterogeneous information network}~\cite{sun2012mining}. A HIN is denoted as $\mathcal{G} = \{\mathcal{V}, \mathcal{E}\}$ consisting of a object set $\mathcal{V}$ and a link set $\mathcal{E}$. A HIN is also associated with an object type mapping function $\phi: \mathcal{V} \rightarrow \mathcal{A}$ and a link type mapping function $\psi: \mathcal{E} \rightarrow \mathcal{R}$. $\mathcal{A}$ and $\mathcal{R}$ denote the sets of predefined object and link types, where $|\mathcal{A}| + |\mathcal{R}| > 2$.
\end{myDef}

The complexity of heterogeneous information network drives us to provide the meta level ($\eg$ schema-level) description for understanding the object types and link types better in the network. Hence, the concept of network schema is proposed to describe the meta structure of a network.
\begin{myDef}
\textbf{Network schema}~\cite{sun2013mining,sun2009ranking}. The network schema is denoted as $\mathcal{S} = (\mathcal{A}, \mathcal{R})$. It is a meta template
for an information network $\mathcal{G} = \{\mathcal{V}, \mathcal{E}\}$ with the object type mapping $\phi: \mathcal{V} \rightarrow \mathcal{A}$ and the link type mapping $\psi: \mathcal{E} \rightarrow \mathcal{R}$, which is a directed graph defined over object types $\mathcal{A}$, with edges as relations from $\mathcal{R}$.
\end{myDef}

\begin{exmp}
As shown in Fig.~\ref{fig_framework}(a), we have represented the setting of movie recommender systems by HINs.
We further present its corresponding network schema in Fig.~\ref{fig_schema}(a), consisting of multiple types of
objects, including User ($U$), Movie ($M$), Director ($D$). There exist different types of links between objects to represent different
relations. A user-user link indicates the friendship between two users, while a user-movie link indicates the rating relation.
Similarly, we present the schematic network schemas for book and business recommender systems in Fig.~\ref{fig_schema}(b) and Fig.~\ref{fig_schema}(c) respectively.
\end{exmp}

In HINs, two objects can be connected via different semantic paths, which are called meta-paths.

\begin{myDef}
\textbf{Meta-path}~\cite{sun2011pathsim}. A meta-path $\rho$ is defined on a network schema $\mathcal{S} = (\mathcal{A}, \mathcal{R})$  and is denoted as a path in the form of $A_1 \xrightarrow{R_1} A_2 \xrightarrow{R_2} \cdots \xrightarrow{R_l} A_{l+1}$ (abbreviated as $A_1A_2 \cdots A_{l+1}$), which describes a composite relation $R = R_1 \circ R_2 \circ \cdots \circ R_l$ between object $A_1$ and $A_{l+1}$, where $\circ$ denotes the composition operator on relations.
\end{myDef}

\begin{exmp}
Taking Fig.~\ref{fig_schema}(a) as an example, two objects can be connected via multiple meta-paths, $\eg$ ``User - User" ($UU$) and  ``User - Movie - User" ($UMU$). Different meta-paths usually convey different semantics. For example, the $UU$ path indicates friendship between two users, while the $UMU$ path indicates the co-watch relation between two users, \ie they have watched the same movies. As will be seen later, the detailed meta-paths used in this work is summarized in Table~\ref{tab_Data}.
\end{exmp}
%Taking Fig. 1(a) as an example, we construct a HIN to model the movie recommendation setting,
% which consists of multiple types of objects (\eg User ($U$), Movie ($M$), Director ($D$)) and links (\eg social relation between users and rating relation between users and movies).
%In this example, two objects can be connected via multiple meta-paths, \eg
% ``User-User" ($UU$) and  ``User-Movie-User" ($UMU$).
% Different meta-paths often convey different semantics. For example, the $UU$ path indicates friendship between two users, while
% the $UMU$ path indicates the two users have watched the same movies. As a major technical approach, meta-path-based search and mining methods have been extensively studied in HINs %\cite{shi2017heterogeneous}.


Recently, HIN has become a mainstream approach to model various complex interaction systems \cite{shi2017survey}. Specially, it has been adopted in recommender systems for characterizing complex and heterogenous recommendation settings.



\begin{myDef}
\textbf{HIN based recommendation}. In a recommender system, various kinds of information can be modeled by a HIN $\mathcal{G} = \{\mathcal{V}, \mathcal{E}\}$. On recommendation-oriented HINs, two kinds of entities (\ie users and items) together with the relations between them (\ie rating relation) are our focus.
Let $\mathcal{U}\subset \mathcal{V}$
and $\mathcal{I}\subset \mathcal{V}$ denote the sets of users and items respectively, a triplet $\langle u, i, r_{u,i}\rangle$  denotes a record that a user $u$ assigns a rating of $r_{u,i}$ to  an item $i$, and $\mathcal{R}=\{\langle u, i, r_{u,i}\rangle\}$ denotes the set of rating records.
We have $\mathcal{U}\subset \mathcal{V}$, $\mathcal{I}\subset \mathcal{V}$ and $\mathcal{R}\subset \mathcal{E}$.
Given the HIN $\mathcal{G} = \{\mathcal{V}, \mathcal{E}\}$, the goal is to predict the rating score $r_{u,i'}$ of $u\in \mathcal{U}$ to a non-rated item $i'\in \mathcal{I}$.
\end{myDef}

Several efforts have been made for HIN based recommendation. Most of these works mainly leverage meta-path based similarities to enhance
the recommendation performance~\cite{yu2013collaborative,yu2014personalized,shi2015semantic,shi2016integrating}. Next, we will present a new heterogeneous network embedding based approach to this task, which is able to effectively
exploit the information reflected in HINs. The notations we will use throughout the article are summarized in Table~\ref{tabl_notations}.

\begin{table}[htbp]
\centering
\caption{Notations and explanations.}\label{tabl_notations}{
\begin{tabular}{c||c}
\hline
{Notation} & {Explanation}\\
\hline
\hline
{$\mathcal{G}$}&{heterogeneous information network}\\
\hline
{$\mathcal{V}$} & {object set} \\
\hline
{$\mathcal{E}$} & {link set} \\
\hline
{$\mathcal{S}$} & {network schema} \\
\hline
{$\mathcal{A}$} & {object type set } \\
\hline
{$\mathcal{R}$} & {link type set} \\
\hline
{$\mathcal{U}$} & {user set} \\
\hline
{$\mathcal{I}$} & {item set} \\
\hline
{$\widehat{r_{u,i}}$} & {predicted rating user $u$ gives to item $i$}\\
\hline
{$\bm{e}_v$} & {low-dimensional representation of node $v$} \\
\hline
{$\mathcal{N}_u$} & {neighborhood of node $u$} \\
\hline
{$\rho$} & {a meta-path} \\
\hline
{$\mathcal{P}$} & {meta-path set} \\
\hline
{$\bm{e}^{(U)}_u, \bm{e}^{(I)}_i$} & {final representations of  user $u$, item $i$}\\
\hline
{$d$} & {dimension of HIN embeddings} \\
\hline
{$D$} & {dimension of latent factors} \\
\hline
{$\mathbf{x}_u, \mathbf{y}_i$} & {latent factors of user $u$, item $i$} \\
\hline
{$\bm{{\gamma}}^{(U)}_u$, $\bm{{\gamma}}_i^{(I)}$} & {latent factors for pairing HIN embedding of user $u$, item $i$} \\
\hline
{$\alpha$, $\beta$} & {parameters for integrating HIN embeddings} \\
\hline
{$\mathbf{M}^{(l)}$} & { transformation matrix w.r.t the $l$-th mete-path} \\
\hline
{$\bm{b}^{(l)}$} & { bias vector w.r.t the $l$-th mete-path} \\
\hline
{$w^{(l)}_u$} & {preference weight of user $u$ over the $l$-th meta-path} \\
\hline
{$\bm{\Theta}^{(U)}$, $\bm{\Theta}^{(I)}$} & {parameters of fusion functions for users, items} \\
\hline
{$\lambda$} & {regularization parameter} \\
\hline
{$\eta$} & {learning rate} \\
\hline
\end{tabular}}
\end{table}
